\documentclass[aspectratio=169, xetex, 12pt]{beamer}
\usepackage[utf8]{inputenc}
\usepackage[T1]{fontenc}
\usepackage{datetime}

\usetheme[mode=dark]{extia}
\title{Dask - Orchestration de tache et mise à l'échelle \\ \Large Monter son projet data de A à Z (en python)}
\author{Clément Dubos}
\institute{Extia}
\newdate{date}{07}{03}{2022}
\date{\displaydate{date}}

\begin{document}

    \maketitle

    \begin{frame}{Sommaire}
        \hfill
        \parbox[t]{.9\textwidth}{
            \begin{minipage}[c][0.2\textheight]{\textwidth}
                \Large
                \tableofcontents
            \end{minipage}
        }
    \end{frame}

    \section{Contexte}

    \begin{frame}{Python, un langage pour la data}{Contexte}
    \end{frame}

    \begin{frame}{Data Array API}{Contexte}
    \end{frame}

    \section{Présentation de Dask}
    \begin{frame}{Dask, c'est quoi ?}{Dask}
    \end{frame}

    \begin{frame}{pourquoi utiliser Dask ?}{Dask}
    \end{frame}

    \section{Démo}
    \begin{frame}{Démo}
    \end{frame}

    \section{Comparaison à PySpark}
    \begin{frame}{Comparaison à PySpark}
    \end{frame}

    \section{Question}
    \begin{frame}{Question}
    \end{frame}

\end{document}
